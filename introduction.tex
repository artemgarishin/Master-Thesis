\chapter{Introduction}
\label{cha:Introduction}

Modern methods to compare of programming pieces of code are used to analyse code's changing, to explore development process and so on. Basically in current tools or plug-ins only text compare methods are used, that is not full sufficient to define code compare.
Sometimes another techniques can be very helpful for such purposes. One of them is a structural code compare, based on building a trees, and methods to compare any similar or same structures. 

TODO START:
You can't write a good introduction until you know what the body of the paper says. Consider writing the introductory section(s) after you have completed the rest of the paper, rather than before.

Be sure to include a hook at the beginning of the introduction. This is a statement of something sufficiently interesting to motivate your reader to read the rest of the paper, it is an important/interesting scientific problem that your paper either solves or addresses. You should draw the reader in and make them want to read the rest of the paper. 

some tips:
A statement of the goal of the paper: why the study was undertaken, or why the paper was written. Do not repeat the abstract. 

TODO STOP: