TODO: review this subsection
\\
The problem of comparing two graphs is known for a long time 
and it is one of the fundamental problems of graph theory.
A. V. Aho, J. E. Hopcroft, and J. D. Ullman present in their book \textit{The Design and
Analysis of Computer Algorithms}  \cite{aho_hopcroft_ulman_algorithms}
algorithms for solving
general problems of subtree isomorphism, largest common subtree and smallest common supertree.
%
S. M. Selkow defined the Top-Down subtree isomorphism \cite{selkow_tree_top_down}.
A definition of  the Bottom-Up subtree isomorphism was introduced by G. Valiente 
in  \cite{valiente_tree_bottom_up_1}, \cite{valiente_tree_bottom_up_2}.
A systematic review of efficient subtree isomorphism algorithms can be found in his book
\textit{Algorithms on Trees and Graphs} \cite{Valiente_algorithms_tree}.
%
The Top-Down and Bottom-Up subtree isomorphism algorithms are used in our proposal
to compare the reduced graphs as trees.
%In our approach control flow graphs have to be transformed into trees.
After the trees have been  compared, the differences are represented 
in graphs, from which the trees have been derived.

There are a number of publications that deal with the comparison of program code.
W. Yang presents an approach  to compare the structure of two programs by using the subtree isomorphism 
%in his  paper \textit{Identifying syntactic differences between two programs} 
\cite{yang_tree_top_down}.
%
Tobias Sager introduced an approach and a tool that measures similarity 
between Java classes \cite{dipl_thesis_sager}.
The tool calculates similarity by using different subtree isomorphism algorithms 
on syntax tree representations of source code.
%
Compared to our proposal, in both approaches syntax trees derived from the source code are used 
instead of control flow graphs. 
The problem is reduced to subtree isomorphism which can be solved in linear time. 
Unfortunately, this approach can not be applied directly for the comparison of two control flow graphs.

%%%%
Laski and Szermer present an algorithm that  finds differences 
of two programs by comparing control flow graphs
\cite{laski_szemer_prog_modification}. 
In their paper they address the problem of the revalidation of a modified code in software maintenance.
The modifications are localized by using the control flow graphs of the original and modified programs. 
Both flow graphs are transformed into reduced flow graphs, 
between which an isomorphic correspondence is calculated.
%
A similar work and a further improvement of the algorithm can be found in \cite{apiwattanapong_orso_jdiff}.
The authors introduced a hammock matching algorithm  based on Laski and Szermer's algorithm.
%
This algorithm uses hammocks and reduced hammock graphs.
Hammocks are subgraphs \cite{ferrante_hamock} which provide a way to impose 
a hierarchical structure on the control flow graph for matching.
%
This approach is very close to our proposal.
In our approach a different type of control flow graph reduction is performed
%The graph is reduced to a tree 
and the comparison is based on subtree isomorphism.
%%%%

Furthermore the problem of comparing the structure of two graphs
can be turned into determining the similarity of two graphs,
which is generally referred to as graph matching problem.
There are a number of publications of graph matching algorithms.
The graph matching has been applied to semantic networks \cite{ehring_semantic_networks}, 
recognition of graphical symbols \cite{jiang_graphical_symbols}, 
three-dimensional object recognition \cite{wong_model_matching}, 
and others.
But the applications and algorithms have a very low relevance to our approach 
because the graphs have always specific properties according to the field of application.
