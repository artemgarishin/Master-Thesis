\chapter{Description of problem}
\label{cha:Description}

In this chapter an issue of the work is being explained. As usual a compare of two codes we can consider them as classes, functions or methods. Thereby a compare can be counted as examinations of two pieces of code, in the best case a methods or functions. They can have a similar implementation or alike syntax, however these two codes are different.

There are many purposes to compare a code, to find out a similarity or determine a difference between them. One of the option is to search for plagiarism in case a code can be taken from external source and a variables have been changed. In addition a general search can be improved to look out a similar code in big projects.

To investigate code comparison, four algorithms of structural compare are required.
To make a contribution into development of structured code compare, the following tasks should be explored:

\begin{enumerate}
  \item The existing algorithms must be investigated (The text-compare method is not sufficient to find a similarity in code)
  \item The algorithms for the structured compare(Abstract syntax trees, Control flow graphs) must be explored 
   \item New methods and algorithms find a place to tried out. A prototypes of combination text-compare and structure-compare can be implemented.
   \item Experimental results of compare must be derived.
\end{enumerate}

The structural compare of trees is not NP-complete problem.