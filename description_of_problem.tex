\chapter{Description of problem}
\label{cha:Description}

In this chapter an issue of the work is being explained. As usual a compare of two codes we can consider them as classes, functions or methods. Thereby a compare can be counted as examinations of two pieces of code, in the best case a methods or functions. They can have a similar implementation or alike syntax, however these two codes are different.

There are many purposes to compare a code, to find out a similarity or determine a difference between them. One of the option is to search for plagiarism in case a code can be taken from external source and a variables have been changed. In addition a general search can be improved to look out a similar code in big projects.

<deploy>

Code is transformed into graphs
there is no deterministic algorithms to compare them because of loops
Graphs - NP complete
The structural compare of trees is not NP-complete problem, it is the reason to compare trees
for trees - yes, graph is into tree transformed

TASK: create a pieces of code and compare them



we need to reduce the problem
we can compare in polynomial time compare trees.

code we transform in tree:
2 questions:
 1. how we transform code in tree
 2. how we compare
 3. how we reference code and nodes(how we put the code difference)
 
Result must be: source code and difference
 
how i can come from code -> graph
2 strategies:

1. AST from source code(briefly explain what it is)	+ experiments
2. 
small example, what kind of result gives text compare(just string compare) and AST

 public void ast1(){
		if(i > 1) i++;
	}
	
	public void ast1(){
		if(i > 1) 
		i++;
	}
	
For byte code:
spanning tree is created so: one edge in the end is deleted(draw how it deleted and why)

can be optimized: byte code has a number

related works(?)
find some works, where nobody was working before(AST optimize)

IDEA: text difference highlight and in parallel nodes in Tree marks and Matching
(TODO: make some experiments, if it's ok, better, worse or same compare)

Statistic + manuel
	
</deploy>

To investigate code comparison, four algorithms of structural compare are required.
To make a contribution into development of structured code compare, the following tasks should be explored:

\begin{enumerate}
  \item The existing algorithms must be investigated (The text-compare method is not sufficient to find a similarity in code)
  \item The algorithms for the structured compare(Abstract syntax trees, Control flow graphs) must be explored 
   \item New methods and algorithms find a place to tried out. A prototypes of combination text-compare and structure-compare can be implemented.
   \item Experimental results of compare must be derived.
\end{enumerate}

