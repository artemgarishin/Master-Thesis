Goal of this work is to search out the most optimal ways to compare different pieces of code. So far there are two techniques of code comparison: a normal text comparison and graph compare. 
Two, similar but not same, pieces of source code must be selected and researched to detect any distinction. These steps can be approached by following:

\begin{enumerate}
  \item Research on source code using existing methods to compare:
  	\begin{enumerate}
   	 	\item normal text compare
   		 \item spanning trees transformed from control flow graphs
	 \end{enumerate}	
  \item Research on source code using existing methods to compare:
  \begin{enumerate}
    \item abstract trees
    \item spanning trees transformed from control flow graphs
  \end{enumerate}
  \item Research on byte code using existing methods to compare:
  \begin{enumerate}
    \item normal text compare
    \item spanning trees transformed from control flow graphs
  \end{enumerate}
\end{enumerate}

	
Play around with patterns of code changing variables, names, sequences of commands and use simple "text to text" compare. In parallel create a control flow graphs and using implemented algorithms TopDown and BottomUp figure out the difference.Compare these both differences and declare received results.
The results can be: 
\begin{enumerate}
  \item same difference
  \item similar difference
  \item full difference
\end{enumerate}
In case similar or full difference a combination or new implementation of algorithms can be further developed.

