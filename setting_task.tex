Goal of this work is to search out the most optimal ways to compare different pieces of code. So far there are two techniques of code comparison: a normal text comparison and graph compare. 
Two pieces of source code, similar but not same, must be selected and researched to detect any distinction. These steps can be approached by following:

\begin{enumerate}
  \item Research on source code using existing methods to compare:
  	\begin{enumerate}
   	 	\item normal text compare
   		 \item spanning trees transformed from control flow graphs
	 \end{enumerate}	
  \item Research on source code using existing methods to compare:
  \begin{enumerate}
    \item abstract trees
    \item spanning trees transformed from control flow graphs
  \end{enumerate}
  \item Research on byte code using existing methods to compare:
  \begin{enumerate}
    \item normal text compare
    \item spanning trees transformed from control flow graphs
  \end{enumerate}
\end{enumerate}

	
Play around with patterns of code changing variables, names, sequences of commands and use simple "text to text" compare. In parallel create a control flow graphs and using implemented algorithms TopDown and BottomUp, figure out the difference. Compare these both differences and declare received results.
The results can be: 
\begin{enumerate}
  \item same difference
  \item similar difference
  \item full difference
\end{enumerate}
In case similar or full difference results, a combination of both methods can be used for optimal comparison. 

Example: we have have two functions that look very similar but nevertheless they have different number of string and different functionality. The abstract results can be following:
\begin{enumerate}
	\item Text compare shows that strings 1 and 5 are different
	\item Graph compare shows that string 7 is different 
\end{enumerate}

Thus combination of two methods can explicit that strings 1,5 and 7 are distinguished and much more distinction has been found.